\documentclass[a4paper,10pt,runningheads]{llncs}

\usepackage{mathrsfs}
\usepackage{stmaryrd}  
% \usepackage{graphicx}
% \usepackage{listings}
% \usepackage{todonotes}
\renewcommand{\sectionmark}[1]{\markright{#1}}

\usepackage{amssymb}
\usepackage{amsmath}
\usepackage{subfigure}
\usepackage{url}
\usepackage{multirow}
\usepackage[ruled, vlined]{algorithm2e}
\usepackage{rotating}
\usepackage{enumitem}

\usepackage[utf8]{inputenc}
\usepackage{xspace}
\usepackage{minted}
\usepackage{hyperref}
\usepackage[authoryear]{natbib}
\usepackage{doi}

\SetAlgorithmName{Strategy}{strategy}{List of Strategies}

\title{Porgy Strategy Language: User Manual}
\author{Maribel Fern\'andez\inst{1} \and H\'el\`ene Kirchner\inst{2} \and Bruno Pinaud\inst{3} \and Jason Vallet\inst{3} }
\institute{King's College London, Department of Informatics, Strand, London WC2R 2LS, UK \email{maribel.fernandez@kcl.ac.uk} \and
Inria, 200 avenue de la Vieille Tour, 33405 Talence, France \email{helene.kirchner@inria.fr} \and
University of Bordeaux, LaBRI CNRS UMR 5800, 33405 Talence Cedex, France \email{firstname.lastname@u-bordeaux.fr}
} 
\newcommand{\urlporgy}{\url{http://porgy.labri.fr/strat_examples}\xspace}
\newcommand{\PORGY}{{\sc Porgy}\xspace}
\newcommand{\TULIP}{{\sc Tulip}\xspace}
\newcommand{\Op}{ \mathcal{O}p}
%\newcommand{\LL}{{\cal L}}
\newcommand{\LL}{Label}
\newcommand{\Interface}{\mathit{Interface}}
\newcommand{\saturated}{\textit{saturated}}
\newcommand{\R}{{\cal R}}
\newcommand{\ppick}[1]{{\tt ppick(}#1{\tt )}}
\newcommand{\all}[1]{{\tt all(}#1{\tt )}}
\newcommand{\one}[1]{{\tt one(}#1{\tt )}}
\newcommand{\CrtGraph}{{\tt crtGraph}}
\newcommand{\CrtPos}{{\tt crtPos}}
\newcommand{\CrtBan}{{\tt crtBan}}
\newcommand{\Id}{{\sf id}}
\newcommand{\Fail}{{\sf fail}}
\newcommand{\isEmpty}[1]{{\tt isEmpty(}#1{\tt )}}
\newcommand{\setPos}[1]{{\tt setPos(}#1{\tt )}}
\newcommand{\setBan}[1]{{\tt setBan(}#1{\tt )}}
\newcommand{\update}[1]{{\tt update(}#1{\tt )}}
\newcommand{\Node}{{\tt node}}
\newcommand{\Edge}{{\tt edge}}
\newcommand{\Port}{{\tt port}}
\newcommand{\tmatch}[1]{{\tt match(}#1{\tt )}}
\newcommand{\orelse}[2]{(#1) {\tt orelse} (#2)}
\newcommand{\whiledoop}[3]{{\tt while(}#1{\tt )}{\tt do(}#2{\tt )}#3}
\newcommand{\whiledoo}[2]{{\tt while(}#1{\tt )do(}#2{\tt )}}
\newcommand{\repeatt}[1]{{\tt repeat(}#1{\tt )}}
\newcommand{\nott}[1]{{\tt not(}#1{\tt )}}
\newcommand{\Property}[3]{{\tt property(}#1, #2, #3{\tt )}}
\newcommand{\Ngb}[3]{{\tt ngb(}#1, #2, #3{\tt )}}
\newcommand{\Ngbopt}[2]{{\tt ngb(}#1, #2{\tt )}}
\newcommand{\try}[1]{{\tt try(}#1{\tt )}}
\newcommand{\ifthenelsee}[3]{{\tt if(}#1{\tt )then(}#2{\tt )else(}#3{\tt )}}
\newcommand{\ifthen}[2]{{\tt if(}#1{\tt )then(}#2{\tt )}}
\newcommand{\set}[2]{{\tt update(}#1\{#2\}{\tt )}}
% \input{../macros}
\begin{document}

\maketitle

\begin{abstract}
% This document provides a complete description of Porgy's strategy language: concrete syntax and semantics illustrated with examples.
% After describing the main constructs in Porgy's strategy language, we provide examples to illustrate its use. 
This document provides concrete syntax illustrated with examples for the \PORGY's strategy
language and the language for rule algorithm and conditions.
\newline
\bf{Last update: \today}
\end{abstract}

% \section{Introduction}
% This document provides the specification  of  \Porgy's strategy language: concrete syntax and semantics illustrated with examples.


% \begin{center}
% \begin{figure}[!ht]
% \includegraphics[width=\textwidth, keepaspectratio]{../images/overview_annotated.png}
%  \caption{Overview of \PORGY: (1) editing one state of the graph being rewritten; (2) editing a rule; (3) some available rewrite rules; (4) portion of the derivation tree, a complete trace of the computing history; (5) the strategy editor.}
%  \label{fig:overview}
% \end{figure}
%  \end{center}

\PORGY is a visual, interactive modelling tool based on port graph rewriting. In \PORGY, system states are represented by port graphs,
and the dynamic evolution of the system is defined via port graph rewrite rules. Strategy expressions are used to control the application of rules, more
precisely, strategy expressions indicate
both the rule to be applied at each step in a rewriting derivation, and the position in the graph where the rule is applied (the latter is done via focusing constructs).

Some of the strategy constructs are
strongly inspired from term rewriting languages such as {\sc
  Elan}~\cite{BorovanskyKKMR98}, Stratego~\cite{Vis01.rta} and
Tom~\cite{TOM-RTA07}.  Focusing operators are not present in term
rewriting languages (although they rely  on implicit traversal
strategies).
% (where the implicit assumption is that the rewrite
% position is defined by traversing the term from the root downwards).x
The direct management of positions in strategy expressions, via the
distinguished position and banned subgraphs %$P$ and $Q$ 
in the target graph and 
%the distinguished graphs $W$, $M$ and $N$ 
in a located port graph rewrite rule
are original features of the language and are managed using
positioning constructs.


This document describes the concrete syntax of strategy expressions, explains how 
 the different kinds of constructs  %expressions 
are used, and provides examples. The complete formal syntax is described in~\cite{fernandezhal-01251871}.% in various application domains. 
 
For more information on \PORGY we refer the reader to~\cite{pinaud:hal-00682550} (interactive features),~\cite{fernandezhal-01251871} for preliminary version of the language and,~\cite{fernandez2018} (social network examples).


\section*{Concrete Syntax for Writing Strategies}\label{syntaxgrammar}
The syntax of the strategy language is given in Table~\ref{tab:syntax-strategies}.
\emph{Strategy expressions} are generated by the grammar rules from
the non-terminal $S$. A strategy expression combines applications of
located rewrite rules, generated by the non-terminal $A$, and position
updates, generated by the non-terminal $U$, using \emph{focusing
  expressions} generated by $F$.  
  
The syntax presented here extends the one in~\cite{FKN:Lopstr} by
including a language to define subgraphs of a given graph by specifying
simple properties, expressed with attributes of nodes, edges and ports.

\begin{center}
\begin{table}[!ht]
\centering
\fbox{
\renewcommand{\arraystretch}{1.2}
\begin{tabular}{l}
 Let $L, R$ be port graphs; 
$M,N$ subgraphs of $R$; $W$ a subgraph of $L$;\\
$n, k\in \mathbb{N}$; 
$\pi_{i=1 \ldots n} \in [0,1]$; $\sum\limits_{i=1}^{n}\pi_i = 1$.
Let \textit{attribute} be an attribute; \\ 
$e$ a valid expression (quoted string, integer or double); \\
\textit{function} a computable function with arguments in
  \textit{parameters\_list}; \\
``script.py'' a Python script which returns the probability
  distribution.\\
$[x]$ means the item $x$ is optional.\\
\hline
\begin{tabular}{lrrrl} 
&{\bf (Comments)} & ~ & ~ & $ /* \ldots */ \mid // \ldots \backslash n$\\
\hline
&{\bf (Probabilities)} & $\Pi$ & $::=$ & $\{\pi_1, \dots, \pi_n\} \mid $ ``script.py'' \\
\hline
\multirow{3}{*}{\begin{sideways}~Rules \end{sideways}}  
  & {\bf (Transformations)} & $T$ & $::=$ & $L_W \Rightarrow_C R_{M}^{N} \mid (T \parallel T) $\\
  & & & $\mid$ & $\ppick{T_1, \dots, T_n, \Pi}$ \\
& {\bf (Applications)} & $A$ & $::=$ & $\all{T} \mid \one{T} $ \\
 \hline
\multirow{5}{*}{\begin{sideways}Positions\end{sideways}} 
& {\bf (Focusing)} &  $F$ & $::=$ & $\CrtGraph \mid  \CrtPos \mid \CrtBan$\\
  &  & & $\mid$ & $F [cup] F \mid  F [cap] F  \mid  F \setminus F\mid (F) \mid  [emptySet] $\\
&  & & $\mid$ & $\ppick{F_1, \dots, F_n, \Pi }$\\
& & & $\mid$ & ${\tt property(}F, Elem [, Expr]{\tt )}$\\
& & & $\mid$ & ${\tt ngb(}F, Elem [, Expr]{\tt )}$\\
 & {\bf (Determining)} & $D$ & $::=$ & $\all{F} \mid \one{F}$\\
& {\bf (Updating)} & $U$ & $::=$ & $ \setPos{D}\mid \setBan{D} $\\
&&& $\mid$ & $\set{function}{parameters\_list}$\\ 
 \hline
\multirow{5}{*}{\begin{sideways}Properties\end{sideways}} & {\bf (Properties)} & $Elem$ & $::=$ &  $\Node \mid \Edge \mid \Port$\\
& & $Expr$& $::=$ &  $  \textit{attribute} ~ Relop ~e \mid Expr \&\& Expr $\\
& & $Relop$ & $::=$ & $ == ~ \mid ~  != ~ \mid ~ > ~ \mid ~ < ~  $\\
& & & $\mid$ & $ =< ~ \mid ~ >= ~ \mid ~ =\thicksim ~$\\
\hline
\multirow{5}{*}{\begin{sideways}Compositions\end{sideways}} 
& {\bf (Comparison)} &  $C$ & $::=$ & $ F = F \mid F !=  F \mid F [subSet] F \mid \isEmpty{F} $\\
& & & $\mid$ & $\tmatch{T}$ \\
& {\bf (Strategies)}   &
 $S$ & $::=$ & $ \Id \mid \Fail \mid A \mid U \mid C \mid S ; S $ \\  
  & & & $\mid$ & ${\tt if(}S{\tt)then(}S{\tt )[else(}S{\tt )]} \mid \orelse{S}{S}$\\
 & & & $\mid$ & $ \repeatt{S}[(k)] \mid \whiledoop{S}{S}{[(k)]}$\\
  & & & $\mid$ & $ \try{S}  \mid \nott{S} \mid \ppick{S_1, \dots, S_n, \Pi} $\\
\end{tabular}
\end{tabular}
}
\caption{Concrete Syntax of the Strategy Language.}\label{tab:syntax-strategies}
\end{table}
\end{center}

We start by defining the necessary syntax to write comments, then Rule constructs, which specify  how to apply
rules, Position constructs, which allow us to
specify subgraphs $P$ and $Q$ in a given located graph. We finally define
Composition constructs combining strategies.

\paragraph{\bf Comments.}
We use C-style comments ($/* \ldots comments \ldots */$) for general multi-line comments and C++-style comments for one line comment ($// \ldots \backslash n$).

\paragraph{\bf Rule Constructs.}

The simplest transformation is a located rule, which can only be
applied to a located graph $G_P^Q$ if at least a part of the redex is
in $P$, and does not involve $Q$.  The syntax $T\parallel T'$
represents simultaneous application of the transformations $T$ and
$T'$ on disjoint subgraphs of $G$; it succeeds if both are possible
\emph{concurrently}, and it fails otherwise.
\begin{itemize}
 \item $\ppick{T_1, \ldots, T_n, \Pi}$ 
picks one of the transformations for application, according to the probability distribution $\Pi$. 
If $T$ and $T'$ have respective probabilities $\pi$ and $\pi'$,
$T\parallel T'$ has probability $\pi \times \pi'$.


\item $\all{T}$ denotes all  possible applications of the transformation $T$ on the
located graph at the current position, creating a new located graph
for each application. In the derivation tree, this creates as many
children as there are possible applications.  

\item $\one{T}$
computes only one of the possible applications of
the transformation and ignores the others; more precisely, it makes a 
choice between all the possible applications, with equal probabilities. 
\end{itemize}



\paragraph{\bf Position Constructs.}

The grammar for $F$ generates focusing
expressions that
are used to define positions for rewriting in a graph, or to define positions where rewriting
is not allowed.  They  denote functions used in strategy expressions to change
subgraphs $P$ and $Q$ in the current located graph $G^P_Q$ (e.g., 
to specify graph traversals).

\begin{itemize}

\item
Focusing constructs  
\begin{itemize}
\item 
$\CrtGraph$, $\CrtPos$ and $\CrtBan$, applied to a located graph
  $G_P^Q$, return respectively the whole graph $G$, $P$ and $Q$.

\item
${\tt property(}F, Elem [, Expr]{\tt \}}$, applied to a located graph
  $G_P^Q$, is used to select elements of $G_P^Q$ filtered by $F$ that
  satisfy a certain property, specified by $Expr$. It can be seen as a
  filtering construct: if the expression $F$ defines a subgraph $G$
  then $\Property{F}{Elem}{Expr}$ returns a subgraph $G'$ of
  $G$ that satisfies the decidable property $Expr$. Depending on the value of $Elem$, the property is
  evaluated on nodes, ports, or edges, allowing us for instance to
  select the red nodes and red edges, or nodes with active ports, as
  shown in examples below. % Note that if an edge is selected, the
  %nodes at each extremities are automatically included in the resulting graph.
  If $Expr$ is not specified, it is considered as the Boolean expression \textit{true} to allow to select all elements indicated by $Elem$.

 \begin{itemize}
  
   \item $\Property{F}{\Node}{Name == ``Add''}$ returns all the nodes of the subgraph defined by the expression $F$ that are named $Add$.

   \item $\Property{F}{\Port}{Active == ``true''}$ returns all the nodes
 of the subgraph defined by the expression $F$ that have at least one port with
an attribute $Active$ with the value $true$.
  
   \item $\Property{F}{\Node}{Colour == Valid}$ returns all the nodes
     of the subgraph defined by the expression  $F$ that have  the
     same values for the attributes  $Colour$ and $Valid$.

 \item $\Property{F}{\Edge}{StateE > 3}$ returns all the edges (including the nodes at their 
extremities) of the subgraph defined by the expression  $F$ that have an attribute  $StateE$ with a 
value greater than $3$.
 
 \item $\Property{F}{\Node}{Name =\thicksim ``\text{\textasciicircum{}Num[0-9]\$}''}$ returns all 
the nodes of the subgraph defined by the expression $F$ with a name valid over the regular 
expression ``{\textasciicircum}Num[0-9]\$'' (the name must start by the string ``Num'' and terminate 
by a number). This syntax is inspired by languages such as Perl, Java or the more recent 
\texttt{C++11}.
 \item $\Property{F}{\Port}{Name == ``Principal'' \&\& State ==``Active''}$ returns 
all nodes having at least one port named ``Principal'' which also has an attribute $State$ 
set to ``Active''.
 
\end{itemize}
  
\item 
${\tt ngb(}F, Elem [, Expr]{\tt \}}$, applied to a located graph
  $G_P^Q$, returns a subset of the neighbours (i.e., adjacent
  nodes) of $F$ according to $Expr$.  When $\Edge$ is used as the element (i.e.,
  when we write $\Ngb{F}{\Edge}{Expr}$), it returns all the
  neighbours of $F$ connected to $F$ via edges which satisfy the
  expression $Expr$.

\begin{itemize}
   \item $\Ngb{F}{\Node}{Name == ``Add''}$, returns all the nodes
     that are adjacent to nodes named $Add$ in $F$ but are not in
     $F$ themselves (i.e., it returns the neighbours of the nodes in
     $F$ named $Add$).


   \item $\Ngb{F}{\Port}{Active == ``true''}$ returns all the nodes
     not already in $F$ that are adjacent to nodes that have a port
     with an attribute  $Active$ set to the value $true$.
  
 \item $\Ngb{F}{\Edge}{State > 3}$ returns the nodes (not already
   in $F$) at the other extremity of edges connected to nodes in $F$,
   where the edge has an attribute $State$ with a value greater than
   $3$.
 
 \item $\Ngbopt{F}{\Edge}$ returns all the nodes adjacent to
   nodes in $F$ and not already in $F$.
\end{itemize}


\item $[cup]$ ($\cup$),  $[cap]$ ($\cap$) and $\setminus$ are union, intersection and
  complement of port graphs; $[emptySet]$ ($\emptyset$) denotes the empty graph.
We assume the usual priorities (e.g., intersection has priority over union) and operations of the same priority are evaluated left to right.

We can combine multiple \textsf{Property} operators using intersection
$\cap$ to filter multiple times.  
For example, to select the nodes in the subgraph denoted
by $Pos$ that are names $Mult$ and that have at least one port named
$Aux$ we write:
\begin{equation*}
 \begin{split}
\all{\Property{Pos}{\Node}{Name == ``Mult''} \cap \\
\Property{Pos}{\Port}{Name == ``Aux''}}  
 \end{split}
\end{equation*}

 

When nodes have more than one port, strategies
\begin{equation*}
 \begin{split}
  \all{\Property{F}{\Port}{Name == ``P''}\cap \\
  \Property{F}{\Port}{State ==``Active''}}
 \end{split}
\end{equation*}
 
     and
$$\all{\Property{F}{\Port}{Name == ``P'' \&\& State ==``Active''}}$$

are not equivalent.   The first strategy returns nodes 
having at least one port named ``P'' and another port (same port or not) with the attribute 
$State$ set to ``Active''. The latter strategy returns nodes having at least one port which 
satisfies both conditions at the same time.

% We give more examples in Section~\ref{Examples}.
     
\item 
 $\ppick{F_1, \dots, F_n,\Pi}$
picks one of the positions for application, according to the given
probability distribution.

\end{itemize}

\item Determine Constructs.\\
$\one{F}$ returns one node in $F$ chosen 
at random and $\all{F}$ returns the full $F$. %When not specified, $F$ stands for $\all{F}$.


\item 
Update Constructs.\\
\begin{itemize}
\item $\setPos{D}$ (resp. $\setBan{D}$) sets the position subgraph $P$
(resp. $Q$) to be the graph resulting from the expression $D$. It
always succeeds (i.e., returns $\Id$).\\
\item $\set{function\_name}{parameters\_list}$ updates attributes and their values 
in the graph using an external \TULIP plugin, with given
parameters. The plugin must be loaded by \TULIP and in the ``\PORGY'' group\footnote{See the \TULIP documentation 
for more information on plugins.}. The 
syntax is the following: $$\set{``plugin\_name''}{param1:value, \ldots, param_n:value}$$
If a parameter is not given, its default value will be used. If the plugin does not have 
parameters using ${\tt update(}``plugin\_name''{\tt)}$ is enough. If a plugin name or a parameter name 
is not valid, an error will be raised and the strategy will not be executed.
This is useful to update global properties of the graph, 
in order to focus on specific nodes. For example, in social networks, 
selecting a ``central'' node.
This is also a way of interfacing with another language (e.g. a Python program or
a plugin written outside \PORGY).
\end{itemize}
\end{itemize}

\paragraph{\bf Composition Constructs.}

The grammar for $S$ involves, beyond previous constructs, an additional class $C$
of comparison constructs, useful for checking conditions.

\begin{itemize}
\item
Comparison constructs:\\ 
$C$ includes comparison operators for graphs
and a matching construct that checks whether a rule matches the
current graph.
\begin{itemize}
 \item 
$ F = F'$  returns $\Id$ if  
both expressions denote isomorphic port graphs (same sets of nodes, ports and edges), otherwise returns 
$\Fail$.  
%MF Questions: do they have to have the same nodes and edges, or nodes and edges with same attributes (same names, etc)???
%BP same nodes and edges. We compate two sets build on one graph
$F ~  !\!\! = ~  F'$ returns $\Id$ if 
the expressions do not denote isomorphic port graphs, otherwise returns $\Fail$.
Similarly $F [subSet] F'$ ($\subset$) checks whether $F$ denotes a subgraph of
$F'$.
We have also included an additional operation, which, although derivable from the rest of the language, facilitates the implementation:
$\isEmpty{F}$ returns $\Id$ if $F$ denotes the empty graph and $\Fail$ otherwise. It is
defined as $ F = emptySet$. 

\item $\tmatch{T}$ returns $\Id$ if there exists a subgraph isomorphism
from the left-hand side of $T$ to the current graph taking into
account the current position and banned subgraphs. In other words,
$\tmatch{T}$ can be seen as an abbreviation of the strategy
$\ifthenelsee{\one{T}}{\Id}{\Fail}$ (see below), but it is directly
implemented to improve its efficiency.
\end{itemize}


\item
Strategies $S$ are  defined with the additional following
constructs:
\begin{itemize}

\item  $\Id$ and $\Fail$ are two atomic strategies that respectively denote
success and failure. 


 \item  The expression $S{\tt ;}S'$ represents sequential application of $S$ followed by $S'$.

 \item ${\tt if(}S{\tt)then(}S'{\tt )[else(}S''{\tt )]}$ checks if the
  application of $S$ on a copy of $G_P^Q$ returns $\Id$, in which case
  $S'$ is applied to the original $G_P^Q$, otherwise $S''$ is applied to the original $G_P^Q$. If $S''$ is not specified then we consider $S''=\Id$.

 \item $\orelse{S}{S'}$  applies $S$ if possible, otherwise
  applies $S'$. It fails if both $S$ and $S'$ fail. 

 \item 

%BP: repeatt and while fixed. No max() operator just a pair of parenthesis. n changed to k to avoid confusion in the grammar (n is used for probabilit distributions)
$\repeatt{S}[(k)]$ simply iterates the application
   of $S$ until it fails, but, if $k$ is specified between parenthesis, then
   the number of repetitions cannot exceed $k$.


 \item $\whiledoo{S}{S'}[(k)]$ keeps on sequentially
   applying $S'$ while the expression $S$ succeeds on a copy of the
   graph.  If $S$ fails, then $\Id$ is returned. If $k$ between parenthesis
   is specified, then the number of iterations cannot exceed $k$.

 \item $\try{S}$ behaves like $S$ if $S$ succeeds, but if $S$ fails,
   it still returns $\Id$. It is a derived operation which is defined
   as $\orelse{S}{\Id}$.
  
\item $\nott{S}$ returns $\Id$ (resp. $\Fail$) if $S$ fails (resp. succeeds). This is also a derivable construct: it is defined as $\ifthenelsee{S}{\Fail}{\Id}$.

\item $\ppick{S_1, \ldots, S_n, \Pi}$
picks one of the stategies for application, according to the given
probability distribution. This 
construct generalises the probabilistic constructs on rules and 
positions seen above.

\end{itemize}

\end{itemize}

% \section{Comments and Macros}
% \label{strat_writing}
% The grammar of the strategy language presented in Table~\ref{tab:syntax-strategies} 
% gives the main syntactic constructs in the language but omits some details, such as the concrete syntax of mathematical 
% operators or how to use the Python feature of ppick. This is given below.

% \subsection{Comments}
% We are using traditional C-style coomment ($/* \ldots comment \ldots */$) for one general multi-line comments and C++-style comment for one line comment ($// \ldots \backslash n$).
% \subsection*{Mathematical Symbols}
% The different mathematical symbols have to be written as described in Tab.~\ref{tab:math}.
% 
% \begin{table}[!ht]
% \centering
%  \begin{tabular}{l|l}
%  \hline
%   Grammar (Tab.~\ref{tab:syntax-strategies})&\Porgy\\
%   \hline
%   $F \cup F$ & $F$ [cup] $F$ \\
%   \hline
%   $F \cap F$ & $F$ [cap] $F$ \\
%   \hline
%   $F \subset F$ & $F [subSet] F$\\
%   \hline
%   $\emptyset$ & $[emptySet]$\\
%   \hline
%  \end{tabular}
% \caption{How to write mathematical symbols inside \porgy}\label{tab:math}
% \end{table}

\section*{Using a Python Script with $\ppick{T_1, \dots, T_n, \Pi}$}
\label{sect:python}

The construct $\ppick{T_1, \dots, T_n, \Pi}$ %(resp. $\ppick{F_1, \dots, F_n, \Pi}$, $\ppick{S_1, \dots, S_n, \Pi}$) => not yet implemented
picks one of the $T$ 
%(resp. $F$, $S$)
operation according to the probability distribution $\Pi$.
$\Pi$ can be given as a list of probabilities, \emph{e.g.}, $\ppick{T_1, \dots, T_n, \{p_1, \dots, p_n\}}$ where 
$\sum_{i=1}^n p_i=1$ or a Python script, \emph{e.g.}, $\ppick{T_1, \dots, T_n, ``script.py``}$ which computes and returns the probabilities.

The basename of the given Python script file (i.e. filename without path information and extension) 
is used as the name of the function to call inside the Python script. The function must have 5 
parameters which are 
in this order: the graph used to apply the rules on (\TULIP graph python object), a list of rules to test (array of strings),
the position subgraph and the banned subgraph (both graphs described by a Boolean property).
The script must return a Python array of strings (\TULIP library does not 
support 
conversion from Python dictionary to C++ object) which has as 
elements the name of a rule followed by its application probability and so on for each rule.  
Note that, all modifications made by the Python script on the graph structure 
are not kept.
A sample Python script is given in Listing~\ref{code:python}. See the \TULIP Python 
documentation for more details on using the \TULIP Python API.

%HK je ne comprends pas le ligne : proba.append(1/len(rules))
\begin{listing}[!ht]
\begin{minted}[linenos,fontsize=\footnotesize,tabsize=2,frame=lines,mathescape]{python}
from tulip import tlp
#toy example to compute a equiprobabilistic choice between all rules
def script(graph, rules, position, ban): 
	#graph: the graph where the rule are applied
	#rules: string array of all rules used in the ppick operator
	#position: Tulip Boolean property for the Position  set
	#ban: Tulip Boolean properties for the Ban set
	#for position and ban, a 'True' indicates the element belongs to the set.
	#proba: a rule name followed by its probability (converted to a string)
	proba=[] 
	for r in rules:
		proba.append(r)
		proba.append(str(1/len(rules)))

	return proba
\end{minted}
\caption{Sample Python3 script to compute probabilities for a ppick operator given a set 
of rules.}
\label{code:python}
\end{listing}

\section*{Macros}
The user interface of \PORGY allows to define multiple strategies. From one strategy, it is possible to call other strategies by surrounding the name of the strategy to use by `\#'. %Strategy~\ref{alg:connected} 
%from example~\ref{sub:graph_testing} can be written as described in Strategy~\ref{alg:macros}.
When the strategy is executed, each string `\#$x$\#' is replaced by the content of $x$. This operation is accessible through a mouse context menu.

% \begin{algorithm}
% \#pick-one-node\#;\\
% \#visit-neighbours\#;\\
% \#check-all-nodes-visited\#\\
% \caption{Strategy to check if a graph is connected using macros. All strategies must exist and be valid. See Strategy~\ref{alg:connected} for the definition of each strategy.}\label{alg:macros}
% \end{algorithm}

% \subsection{Update construct.}
% $\set{function\_name}{parameters\_list}$ updates attributes and their values 
% in the graph using an external Tulip plugin, with given
% parameters. The plugin must be loaded by \Tulip and must be in the ``Porgy'' group\footnote{See the \Tulip documentation 
% for more information on plugins.}. The 
% syntax is the following: $$\set{``plugin\_name''}{param1:value, param2:value, \ldots, 
% param_n:value}$$
% If a parameter is not given, its default value will be used. If the plugin does not have 
% parameters using ${\tt update(}``plugin\_name''{\tt)}$ is enough. If a plugin name or a parameter name 
% is not valid, an error will be raised and the strategy will not be executed.

\section*{Concrete Syntax for Computing Attributes Values}\label{app:inner}
The value of any attribute of any element (nodes or ports) in the right-hand side can be computed with a formula.
% In example~\ref{sub:propagation}, the value of $Sigma$ in rule $R1$ (Fig.~\ref{fig:LT_rules}) is given by Eq.~\ref{eq:sigma}.
%In \Porgy, the user interface to write
There is a dedicated user interface in the ``Algorithm'' configuration tab associated to each 
rule. All formulas associated to a rule are stored into each rule and are separated by a semi-colon.
Table~\ref{table:syntax-inner} gives the 
associated grammar. 
%At the time of writing, 
Up to now, we only support 
expressions for attributes of type double or integer without mixing them. 

\label{app:inner_rule_exp}
\begin{center}

\begin{table}[!ht]
\centering
\fbox{
\renewcommand{\arraystretch}{1.5}
\begin{tabular}{l}

Let $r\in \mathbb{R}$; Let $L=(V_L, P_L, E_L, D_L)$ and \\
$R=(V_R, P_R, E_R, D_R)$ be port graphs (resp. left and right-hand side);\\
with $V$ the set of nodes, $P$ the set of ports, $E$ the set of edges,\\
$D$ the associated record;\\
$n_L \in V_L$, $e_L \in E_L$, $p_L \in P_L$ (resp. $n_R$, $e_R$ and $p_R$ for $R$); \\
\textit{attribute} be an existing attribute.\\ 
$\Omega_{L}$ (resp. $\Omega_{R}$) generates attribute values in records $D$\\
associated with $L$ (resp. $R$);\\
$[S]$ means the item $S$ is optional.\\
\hline
\begin{tabular}{l|rrl}
   {\bf (Expression)} & $S$ & $::=$ & $\Omega_{R} = \Psi ; [S]$ \\
%HK could we add parentheses? for instance write :
% {\bf (Expression)} & $S$ & $::=$ & $(\Omega_{R} := \Psi ) ; [S]$ \\
   \hline
{\bf (Attribute access)} & $\Omega_{R}$ & $::=$ &  ${\tt node}( n_R ).\textit{attribute} \mid{\tt 
edge}(e_R).\textit{attribute}$\\
  & & $\mid$ & ${\tt port}( p_R ).\textit{attribute}$\\
  & $\Omega_L$ & $::=$ &  ${\tt node}( n_L {\tt ) }.\textit{attribute} \mid {\tt 
 edge}(e_L {\tt )}.\textit{attribute}$\\
  & & $\mid$ & ${\tt port}( p_L {\tt)}.\textit{attribute}$\\
 \hline
  {\bf (Instruction)} &  $\Psi$ & $::=$ & $r \mid \Omega_{L} \mid ( \Psi ) \mid -\Psi \mid \Psi ~ op ~ \Psi$ \\
    & & $\mid$ & ${\tt random}( r ) \mid {\tt max}(\Psi, \Psi) \mid {\tt min}(\Psi, \Psi)$\\
    & $op$ & $::=$ & $ + \mid - \mid \times \mid / \mid \%$\\
\end{tabular}
\end{tabular}
}
\caption{Grammar for rule formulas}\label{table:syntax-inner}
\end{table}
\end{center}

For example, to compute a new value for an attribute called {\tt Sigma} of a node $n$ where the value is a maximum between the old value and a ratio beetween the value of another attribute over a random number is written: 
\[ 
node(n).Sigma = max(edge(e).Influence/random(1),node(n).Sigma);
\]

where $n$ and $e$ (and if necessary $p$) are the internal \TULIP element identifier. This 
information is available through the ``Get Information'' interactor.

An {\bf Instruction} always returns a double number and  follows the standard 
mathematical priority. 
The construct {\tt random(r)} returns a double number $0 < x < r$; {\tt max(x, y)} (resp. 
{\tt min(x,y)}) returns the maximum (resp. minimum) of both arguments.

\bibliographystyle{rnti}
\bibliography{bib}

\end{document}
